\documentclass[aspectratio=169, 10pt]{beamer}

% --- Load the Bristol Theme ---
\usepackage{./template/flight_lab_theme} 

% --- Standard Packages ---
\usepackage[utf8]{inputenc}
\usepackage[T1]{fontenc}
\usepackage{caption}

% --- Presentation Info ---
\title{Presentation Title}
\author{Firstname Surname}
\institute{University of Bristol} 
\date{12/03/25}

\begin{document}

% --- Slide 1: Title ---
{
% We hide the footer here because the title page template handles it
\setbeamertemplate{footline}{} 
\begin{frame}
    \titlepage
\end{frame}
}

% --- Slide 2: About Me ---
\begin{frame}
    \frametitle{Bullets and Spacing}
    Here are some interesting facts about me:
    
    \begin{itemize}
        \item Fact one
        \item Fact two
        \item Fact three
    \end{itemize}
\end{frame}

% --- Slide 3: Block & Footnote Test ---
\begin{frame}{Blocks and Footnotes}

    \begin{block}{Observation 1}
        Here is an extremely interesting observation in red.\footnote{This is a footnote.}
    \end{block}

    \begin{exampleblock}{Observation 2}
        Here is another extremely interesting observation in green.
    \end{exampleblock}

    \begin{navyblock}{Observation 3}
        Here is yet another extremely interesting observation in blue.
    \end{navyblock}
\end{frame}

\begin{frame}{Equations}
Here is a nonsensical equation:
    \begin{equation}
        \int_{x\in \mathcal{X}\oplus\Omega(t)} x(\tau) d\tau
    \end{equation}
\end{frame}

\begin{frame}{Images}
    \begin{columns}
        \column{0.5\textwidth}
        \centering
        \includegraphics[width=0.8\linewidth]{puppy.jpeg}
        \captionof{figure}{Puppy 1}

        \column{0.5\textwidth}
        \centering
        \includegraphics[width=0.8\linewidth]{puppy.jpeg}
        \captionof{figure}{Puppy 2}
    \end{columns}
\end{frame}

\begin{frame}{Tables}
    \begin{table}
        \centering
        \begin{tabular}{|c|c|c|}
            \hline
            \textbf{Header 1} & \textbf{Header 2} & \textbf{Header 3} \\
            \hline
            Row 1, Col 1 & Row 1, Col 2 & Row 1, Col 3 \\
            \hline
            Row 2, Col 1 & Row 2, Col 2 & Row 2, Col 3 \\
            \hline
            \textbf{Total} & \textbf{100} & \textbf{Yes} \\
            \hline
        \end{tabular}
        \caption{A sample table description.}
    \end{table}
\end{frame}

\end{document}